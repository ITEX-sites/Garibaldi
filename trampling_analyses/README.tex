% Options for packages loaded elsewhere
\PassOptionsToPackage{unicode}{hyperref}
\PassOptionsToPackage{hyphens}{url}
%
\documentclass[
]{article}
\usepackage{amsmath,amssymb}
\usepackage{iftex}
\ifPDFTeX
  \usepackage[T1]{fontenc}
  \usepackage[utf8]{inputenc}
  \usepackage{textcomp} % provide euro and other symbols
\else % if luatex or xetex
  \usepackage{unicode-math} % this also loads fontspec
  \defaultfontfeatures{Scale=MatchLowercase}
  \defaultfontfeatures[\rmfamily]{Ligatures=TeX,Scale=1}
\fi
\usepackage{lmodern}
\ifPDFTeX\else
  % xetex/luatex font selection
\fi
% Use upquote if available, for straight quotes in verbatim environments
\IfFileExists{upquote.sty}{\usepackage{upquote}}{}
\IfFileExists{microtype.sty}{% use microtype if available
  \usepackage[]{microtype}
  \UseMicrotypeSet[protrusion]{basicmath} % disable protrusion for tt fonts
}{}
\makeatletter
\@ifundefined{KOMAClassName}{% if non-KOMA class
  \IfFileExists{parskip.sty}{%
    \usepackage{parskip}
  }{% else
    \setlength{\parindent}{0pt}
    \setlength{\parskip}{6pt plus 2pt minus 1pt}}
}{% if KOMA class
  \KOMAoptions{parskip=half}}
\makeatother
\usepackage{xcolor}
\usepackage[margin=1in]{geometry}
\usepackage{graphicx}
\makeatletter
\def\maxwidth{\ifdim\Gin@nat@width>\linewidth\linewidth\else\Gin@nat@width\fi}
\def\maxheight{\ifdim\Gin@nat@height>\textheight\textheight\else\Gin@nat@height\fi}
\makeatother
% Scale images if necessary, so that they will not overflow the page
% margins by default, and it is still possible to overwrite the defaults
% using explicit options in \includegraphics[width, height, ...]{}
\setkeys{Gin}{width=\maxwidth,height=\maxheight,keepaspectratio}
% Set default figure placement to htbp
\makeatletter
\def\fps@figure{htbp}
\makeatother
\setlength{\emergencystretch}{3em} % prevent overfull lines
\providecommand{\tightlist}{%
  \setlength{\itemsep}{0pt}\setlength{\parskip}{0pt}}
\setcounter{secnumdepth}{-\maxdimen} % remove section numbering
\ifLuaTeX
  \usepackage{selnolig}  % disable illegal ligatures
\fi
\IfFileExists{bookmark.sty}{\usepackage{bookmark}}{\usepackage{hyperref}}
\IfFileExists{xurl.sty}{\usepackage{xurl}}{} % add URL line breaks if available
\urlstyle{same}
\hypersetup{
  hidelinks,
  pdfcreator={LaTeX via pandoc}}

\author{}
\date{\vspace{-2.5em}}

\begin{document}

\hypertarget{trampling-analyses}{%
\subsection{TRAMPLING ANALYSES}\label{trampling-analyses}}

Tests i) effects of human trampling disturbance on plant growth and
reproduction proxies of common plants across an elevation gradient, and
ii) which species are more susceptible to trampling.

\hypertarget{contributors}{%
\subsection{Contributors:}\label{contributors}}

Nathalie Chardon
(\href{mailto:nathalie.chardon@gmail.com}{\nolinkurl{nathalie.chardon@gmail.com}}),
Carly Hilbert
(\href{mailto:chilbe02@student.ubc.ca}{\nolinkurl{chilbe02@student.ubc.ca}}),
Philippa Stone
(\href{mailto:philippa.stone@botany.ubc.ca}{\nolinkurl{philippa.stone@botany.ubc.ca}}),
Cassandra Elphinstone
(\href{mailto:cassandra.elphinstone@shaw.ca}{\nolinkurl{cassandra.elphinstone@shaw.ca}}),
Allen Zhao
(\href{mailto:allen10to11@gmail.com}{\nolinkurl{allen10to11@gmail.com}})

Last edited: 14 June 2023

\#----------------------------- \#\# Data collection notes summer 2022 -
Who, when, where, why and how the data were collected

Dates: July-August 2022

Location: Garibaldi Provincial Park on Taylor Meadow (TM), Black Tusk
(BT), and Panorama Ridge (PR) trails.

Samplers: Nathalie Chardon, Carly Hilbert, Mackenzie Urquhart-Cronish,
Brianna Ragsdale, Teagan MacLachlan, Vickie Lee, Christian Lauber,
Carolyn Chong

Data entered by: Carly Hilbert

Methods: In the summer of 2022, we established long-term transects near
the major trails in the park (Taylor Meadows, Black Tusk, Panorama
Ridge) to quantify the effects of trampling by recreational users. To
address how trampling affects charismatic plant communities (blueberry,
heather and sedge meadows) along elevational gradients, we chose sites
at multiple elevations per trail. We established transects directly
adjacent to the trail and at least a 5 m perpendicular distance away
from the trail (control) to compare the effects of trampling on the same
vegetation types.

Plant communities. We recorded landscape characteristics (slope, aspect,
latitude, longitude, presence of trees) for each transect and used 0.5m
x 1m quadrats to record height, maximum diameter, and bud/flower/fruit
counts of our focal plant species (Vaccinium ovalifolium, Cassiope
mertensiana, Phyllodoce empetriformis, Phyllodoce grandiflora, Carex
sp.). Because we only found Phyllodoce grandiflora at one site, we did
not use this species in our analyses. Please see N. Chardon's recent
work on human trampling for greater detail on this sampling approach:
\url{https://besjournals.onlinelibrary.wiley.com/doi/full/10.1111/1365-2664.13384}
\url{https://onlinelibrary.wiley.com/doi/full/10.1002/ece3.4276}

We also used a standardized approach to photograph each quadrat, and are
used these images to calculate how vegetation greenness is impacted by
human trampling. We calcluated plant percent cover with a custom
algorithm.

Microbial communities. We selected representative transects per trail to
study microbial activity. As a proxy for measuring litter decomposition
by microbes, we buried tea bags at these transects to undergo microbial
decomposition for one year.

Upcoming work. In Summer 2023, we will return to each transect to
re-survey transects for species diversity. We will dig out the buried
teabags, and bury new ones to decompose for another year. The species
diversity and microbial data are not currently used in analyses.

Goals: We aim to quantify the continued effects of human trampling on
these plant and microbial communities by returning each year, thus
generating a multi year dataset. Such a dataset will allow us to answer
how these plant communities are responding to continuous trampling and
ambient warming. We will incorporate BC Parks data on visitation numbers
in our analyses to answer how strongly yearly visitation rates correlate
with community responses, or if these responses show more of a lag
response to past visitation rates.

\#----------------------------- \#\# Data files \#\# Raw data FILENAME
\textless description of file contents, date collected\textgreater{}

Trampling-TRANSECTS\_data.csv/xlsx: 10 m x 0.5 m transect-level data
(see merge\_fielddata.R for abbreviations); July-Aug 2022

Trampling-QUADS\_data.csv/xlsx: quad-level data (see merge\_fielddata.R
for abbreviations); July-Aug 2022

GPS\_with-elev: lat, long, elev for each transect (see
merge\_fielddata.R for data details); July-Aug 2022

\hypertarget{compiled-data}{%
\subsection{Compiled data}\label{compiled-data}}

FILENAME \textless description of file contents, date modified, whether
mid or final version\textgreater{}

quad.RData: gps \& transect data matched to quad data; created in
merge\_fielddata.R; modifed on 25 Apr 2023; final version; variables
described in merge\_fielddata.R

trans\_ALL.RData: all transect field data, gps, and altitude; created on
15 Nov 2022; final version; variables described in merge\_fielddata.R

P\_albicaulis\_Garibaldi\_Aug2022.csv: Pinus albicaulis locations for BC
Rangers; emailed data to Kym Welstead
(\href{mailto:Kym.Welstead@gov.bc.ca}{\nolinkurl{Kym.Welstead@gov.bc.ca}})
on 7 Dec 2022

plant-percent-cover.csv: Estimated plant percent cover values for each
quadrat (Allen Zhao:
\href{mailto:allen10to11@gmail.com}{\nolinkurl{allen10to11@gmail.com}});
created on 27 March 2023; mid version because many transects missing and
data needs to be checked

\#----------------------------- \#\# Analysis notes FILENAME
\textless description of notes purpose, coding language, versions of
software/libraries used\textgreater{}

\#----------------------------- \#\# Scripts SCRIPTNAME
\textless description of script purpose, coding language, versions of
software/libraries used\textgreater{}

merge\_fielddata.R: merge GPS, elevation, transect, and quad-level data;
R; tidyverse, dplyr, lattice

LMMs.R: basic linear mixed models testing the effects of trampling and
elevation on plant responses; R; tidyverse, lmerTest, AICmodavg

figs.R: figures visualizing data; R; ggplot2, ggsignif, tidyr, dplyr

repro.R: define and calculate a standardized reproductive metric, add to
quads.RData, and test relationship between reproductive density and
plant area by species; ggplot2, ggtext, scales, tidyverse, brms

\#----------------------------- \#\# Figures FOLDERNAME

/publication/: figures for publication (Mar 2023)

/BC\_PARF/: Carly's figures for BC PARF (Dec 2022)

\end{document}
